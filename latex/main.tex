

\documentclass[conference,compsoc]{IEEEtran}


\usepackage{graphicx}
\usepackage{amsmath}
\usepackage{amssymb}
\usepackage{amsfonts}
\usepackage{algorithm}
\usepackage{algpseudocode}
\usepackage{url}
\usepackage{multirow}
\usepackage{dcolumn}
\usepackage{subfig}
\usepackage{listings}
\usepackage{color}
\usepackage{siunitx}
\usepackage{subfig}


\ifCLASSOPTIONcompsoc
  % IEEE Computer Society needs nocompress option
  % requires cite.sty v4.0 or later (November 2003)
  \usepackage[nocompress]{cite}
\else
  % normal IEEE
  \usepackage{cite}
\fi

% correct bad hyphenation here
\hyphenation{op-tical net-works semi-conduc-tor}

\begin{document}
%

\title{Aplicação de Modelo de Rede Neural Convolucional EEGNet em Dados Eletroencéfalográficos de Imagética Motora}


% author names and affiliations
% use a multiple column layout for up to three different
% affiliations
\author{Seidi Y. Yamauti\\
\IEEEauthorblockA{IIN-ELS \\
Email: seidi.yamauti@edu.isd.org.br}
}

% make the title area
\maketitle

% As a general rule, do not put math, special symbols or citations
% in the abstract
\begin{abstract}

O modelo de rede neural convolucional EEGNet foi testado com dados EEG de imagética motora adquiridos através da plataforma EEGBCI. Os resultados de classificação foram comparados com modelos consolidados xDAWN + RG e CSP + LDA, obtendo a mesma acurácia. O respositório deste estudo está disponível em: \url{https://github.com/seidikun/trabalho_final_progamacao}

\end{abstract}

% no keywords

% creates the second title. It will be ignored for other modes.
\IEEEpeerreviewmaketitle


\section{Introduction}

Interfaces-Cérebro Máquina (ICMs) são uma solução em neuroengenharia que oferece comunicação e capacidades de controle de dispositivos eletromecânicos para pessoas com deficiências motoras severas \cite{nicolas2012brain}. Estudos atuais têm utilizado interfaces de propósito-único e que não generalizam para mais que uma tarefa. Por outro lado, o uso de aprendizado profundo (Deep Learning) oferece a possibilidade de desenvolvimento de algoritmos que consigam estabelecer um aprendizado que abranja diversas tarefas em um único modelo \cite{lecun2015deep} e por isso é interessante que haja maior intersecção entre as áreas de ICM e aprendizado de máquina


\section{Materiais e Métodos}

\subsection{EEGNet}

O modelo EEGNet é uma rede neural convolucional proposto em \cite{lawhern2018eegnet} e desenvolvido com 3 propriedades:

\begin{enumerate}
    \item Ser aplicável em diversos paradigmas de ICM 
    \item Pode ser treinado com uma quantidade limitada de dados
    \item Produz características fisiológicamente interpretáveis
\end{enumerate}

O modelo está disponível através da plataforma GitHub no link \url{https://github.com/vlawhern/arl-eegmodels} 

\subsection{Dados EEGBCI}

Uma plataforma para estudos de ICM para propósitos gerais foi desenvolvida em \cite{schalk2004bci2000}, sendo capaz de combinar diversos tipos de sinais cerebrais, métodos de processamento de sinais, dispositivos de saída e protocolos operacionais. Os dados utilizados no presente estudos foram extraídos através desta plataforma e estão disponíveis pelo banco de dados PhysioNet \cite{goldberger2000physiobank}

Os dados utilizados foram obtidos do sujeito com id 93, blocos 4,8 e 12, conforme a descrição dos dados para tentativas de imagética motora de mão direita e mão esquerda. O pré-processamento é realizado por meio de filtro digital passa-bandas por janelamento (firwin), com frequêcia mínima de corte 7Hz e frequência máxima de corte 30Hz. Os dados filtrados EEG são então janelados em torno dos eventos marcados, com 1 segundo antes e 4 segundos após a marcação no sinal, de onde se obtem 45 janelas, sendo 23 de imagética motora utilizando a mão esquerda e 22 de imagética motora utilizando a mão direita.

\subsection{Algoritmos Consolidados de Classificação de Imagética Motora em dados EEG}

Para efeito de comparação com o modelo EEGNet, dois algoritmos foram utilizados para classificação de imagética motora em EEG e suas respectivas acurácias extraídas: o modelo de filtro xDAWN com aplicação de geometria riemanniana é proposta em \cite{rivet2009xdawn} \cite{congedo2013new} e obteve o primeiro lugar na competição Kaggle BCI \url{http://github.com/alexandrebarachant/bci-challenge-ner-2015} para detecção de potenciais erros de aquisição de dados em uma tarefa de soletramento usando o paradigma P300 em ICM e foi utilizado para comparação com a EEGNet em \cite{lawhern2018eegnet}. Já o modelo CSP + LDA \cite{koles1991quantitative} é comumente utilizado para classificação de imagética motora em dados EEG

\section{Resultados}

Os três modelos obtiveram acurácia de 100\% sobre os dados EEGBCI de imagética motora. O modelo EEGNet foi testado com diversas configurações, demonstrando instabilidade sobre os dados, o que será discutido na seção \ref{discussion}

\begin{figure}[hbt!]
 \centering
 \includegraphics{cm_lda}
\end{figure}

\section{Discussão} \label{discussion}
O modelo EEGNet apresenta a perspectiva de aprendizado de máquina profundo, de forma a substituir o modelo clássico que envolve especificação de extração de características e especificação de classificação. Nesta perspectiva, as duas fases estão contidas em um único modelo, possiblitando a atualização de parâmetros de forma genérica.
No estudo com dados de imagética motora, a EEGNet obteve a mesma acurácia que modelos clássicos, evidenciado como o aprendizado de máquina profundo pode conter as especificações de generalidade e alta performance. Porém, foi notado que a acurácia de prediçao da EEGNet é sensível aos valores de parâmetros de definição, necessitando de um estudo mais aprofundado sobre sua explicabilidade

\section{Conclusão}

O modelo EEGNet foi aplicado em dados EEG de imagética motora com os mesmos parâmetros citados em \cite{lawhern2018eegnet}, obtendo resultado comparável a modelos consolidados para a classificação do mesmo tipo de paradigma em ICM. Estudos futuros podem incluir a explicabilidade do modelo em relação a caraxterísticas fisiologicamente prováveis e estudo com dados mais estensos, envolvendo mais sujeitos e protocolos de treino diversos.
\bibliographystyle{IEEEtran}
\bibliography{references}

\end{document}

