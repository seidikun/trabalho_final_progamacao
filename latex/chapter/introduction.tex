\chapter[Introdução]{Introdução}

\textbf{Ler com bastante atenção:}

Alguns \textbf{manuais}, \textbf{exemplos} e \textbf{vídeos} de pacotes latex foram adicionados no capítulo \textbf{\ref{cap:cap01}}.

Todo o documento já está formatado com o layout utilizado pelo IIN-ELS. Você precisa apenas inserir as informações do seu trabalho nos respectivos tópicos.
É importante lembrar que todos os capítulos, subcapítulos e suas seções contidas aqui, são apenas exemplos, não sendo obrigatório para a sua qualificação/dissertação. Confiram os itens obrigatórios no documento "Normativa do Exame de Qualificação do Mestrado Acadêmico em Neuroengenharia".

Procure se informar a respeito do Mendeley. É um software para gerenciar referencias bibliográficas que irá facilitar muito a inclusão das referências no documento. Procure também como importar as referencias inseridas no mendeley para o arquivo references.bib aqui no overleaf.

Este documento e seu código-fonte são exemplos de referência de uso da classe \textsf{abntex2} e do pacote \textsf{abntex2cite}. O documento exemplifica a elaboração de trabalho acadêmico (tese, dissertação e outros do gênero) produzido conforme a ABNT NBR 14724:2011 \emph{Informação e documentação - Trabalhos acadêmicos - Apresentação}.

A expressão ``Modelo Canônico'' é utilizada para indicar que \abnTeX\ não é modelo específico de nenhuma universidade ou instituição, mas que implementa tão somente os requisitos das normas da ABNT. Uma lista completa das normas observadas pelo \abnTeX\ é apresentada em \citeonline{abntex2classe}. 

Este documento deve ser utilizado como complemento dos manuais do \abnTeX\ \cite{abntex2classe,abntex2cite,abntex2cite-alf} e da classe \textsf{memoir} \cite{memoir}.

%pode apagar!
Para alterar este texto basta dar um 
\textbf{duplo clique “aqui”}, e o local de edição será aberto na sua caixa de edição do lado “esquerdo”.

%pode apagar!
\textbf{Copie e cole seu texto aqui.}


\section{Justificativa}
	%pode apagar!
Para alterar este texto basta dar um 
\textbf{duplo clique “aqui”}, e o local de edição será aberto na sua caixa de edição do lado “esquerdo”.

%pode apagar!
\textbf{Copie e cole seu texto aqui.}


\section{Delimitação do tema ou problema}
    %pode apagar!
Para alterar este texto basta dar um 
\textbf{duplo clique “aqui”}, e o local de edição será aberto na sua caixa de edição do lado “esquerdo”.

%pode apagar!
\textbf{Copie e cole seu texto aqui.}


\section{Delimitação do trabalho}

%pode apagar!
Para alterar este texto basta dar um 
\textbf{duplo clique “aqui”}, e o local de edição será aberto na sua caixa de edição do lado “esquerdo”.

%pode apagar!
\textbf{Copie e cole seu texto aqui.}

\section{Objetivos}

%pode apagar!
Para alterar este texto basta dar um 
\textbf{duplo clique “aqui”}, e o local de edição será aberto na sua caixa de edição do lado “esquerdo”.

%pode apagar!
\textbf{Copie e cole seu texto aqui.}

\subsection{Objetivos Gerais}
	
	%pode apagar!
Para alterar este texto basta dar um 
\textbf{duplo clique “aqui”}, e o local de edição será aberto na sua caixa de edição do lado “esquerdo”.

%pode apagar!
\textbf{Copie e cole seu texto aqui.}

    
\subsection{Objetivos Específicos}
    %pode apagar!
Para alterar este texto basta dar um 
\textbf{duplo clique “aqui”}, e o local de edição será aberto na sua caixa de edição do lado “esquerdo”.

%pode apagar!
\textbf{Copie e cole seu texto aqui.}
