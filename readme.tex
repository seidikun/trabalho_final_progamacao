% Este documento tem o objetivo de explicar minimamente como realizar as modificações necessárias para que o texto da dissertação seja corretamente formatado de acordo com os padrões do programa de pós-graduação em neuroengenharia.

%A pasta é organizada da seguinte maneira:
%   * Pretextual - nesta pasta estão contidos os arquivos relacionados aos resumo, agradecimentos, capa, dedicatória, epígrafe, ficha catalográfica, folha de aprovação, listas e sumário.
%   * Figuras - todas as figuras do documento estarão contidas na pasta figuras.
%   * chapter - nesta pasta devem estar contidos todos os arquivos que compõe o texto da dissertação como a introdução,  referencial teórico, metodologia, resultados e discussão e conclusão.
%   * postextuais - nesta pasta devem estar contidos os arquivos dos elementos pós-textuais, ou seja, apêndices e anexos.

% O arquivo mais importante deste conjunto de arquivos que formam a dissertação é o "main.tex". Este arquivo é responsável por concentrar todos os outros e permitir que a compilação seja realizada adequadamente. Caso haja a necessidade de o usuário adicionar novos elemetos pré, pós ou textuais, eles devem ser incluídos na sequência correta dentro do arquivo main.tex. Uma observação importante a ser realizada é que não há a necessidade de modificar o arquivo main exceto que o usuário deseje adicionar novos arquivos ao texto.

% Algumas considerações adicionais são importantes ao trabalhar com textos feitos em Latex, e, por isso, seria interessante fundamentar o conhecimento com alguns assuntos básicos a respeito da padronização de arquivos com o Latex:
%   * Importância do Latex: https://www.youtube.com/watch?v=QCEqv7wMmIg
%   * Criando o primeiro Texto: https://www.youtube.com/watch?v=Bivw_raoaz4
%   * Segmentando o texto em arquivos: https://www.youtube.com/watch?v=LXWOazgNqK4
%   * Criando sistema de bibliografias: https://www.youtube.com/watch?v=oztGf4m9vl8
%   * Formatando figuras: https://www.youtube.com/watch?v=0Lesgsa6zLw
%   * Criando Tabelas: https://www.youtube.com/watch?v=Q6H67XU1zuc
%   * Criando Equações: https://www.youtube.com/watch?v=aH-GmAZSoSg
%   * Criando itens e enumeração: https://www.youtube.com/watch?v=Ibj9wqCNNAg

% É importante lembrar que as explicações de cada tema levam em consideração a sua aplicação imediata em um documento com formatação básica. Porém, todo o conteúdo pode ser aplicado conceitualmente ao documento atual e, no caso das figuras, tabelas, equações e itens, o conteúdo pode e deve ser aplicado em cada arquivo dos elementos de texto contidos neste projeto.