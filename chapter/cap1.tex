\chapter{Exemplo de Capítulo}
\label{cap:cap01}
%pode apagar!
Para alterar este texto basta dar um 
\textbf{duplo clique “aqui”}, e o local de edição será aberto na sua caixa de edição do lado “esquerdo”.

%pode apagar tudo!
\textbf{Copie e cole seu texto aqui.}

* Pretextual - nesta pasta estão contidos os arquivos relacionados aos resumo, agradecimentos, capa, dedicatória, epígrafe, ficha catalográfica, folha de aprovação, listas e sumário.

* figuras - todas as figuras do documento estarão contidas na pasta figuras.

* chapter - nesta pasta devem estar contidos todos os arquivos que compõe o texto da dissertação como a introdução,  referencial teórico, metodologia, resultados, discussão e conclusão.

* postextuais - nesta pasta devem estar contidos os arquivos dos elementos pós-textuais, ou seja, apêndices e anexos.

 O arquivo mais importante deste conjunto de arquivos que formam a dissertação é o "main.tex". Este arquivo é responsável por concentrar todos os outros e permitir que a compilação seja realizada adequadamente. Caso haja a necessidade do usuário adicionar novos elemetos pré, pós ou textuais, eles devem ser incluídos na sequência correta dentro do arquivo main.tex. Uma observação importante a ser realizada é que não há a necessidade de modificar o arquivo main exceto que o usuário deseje adicionar novos arquivos ao texto.

 Algumas considerações adicionais são importantes ao trabalhar com textos feitos em Latex, e, por isso, seria interessante fundamentar o conhecimento com alguns assuntos básicos a respeito da padronização de arquivos com o Latex:
 
 * Importância do Latex: \url{https://www.youtube.com/watch?v=QCEqv7wMmIg}

* Criando o primeiro Texto: \url{https://www.youtube.com/watch?v=Bivw_raoaz4}

* Segmentando o texto em arquivos: \url{https://www.youtube.com/watch?v=LXWOazgNqK4}

* Criando sistema de bibliografias: \url{https://www.youtube.com/watch?v=oztGf4m9vl8}

* Formatando figuras: \url{https://www.youtube.com/watch?v=0Lesgsa6zLw}

* Criando Tabelas: \url{https://www.youtube.com/watch?v=Q6H67XU1zuc}

* Criando Equações: \url{https://www.youtube.com/watch?v=aH-GmAZSoSg}

* Criando itens e enumeração: \url{https://www.youtube.com/watch?v=Ibj9wqCNNAg}

É importante lembrar que as explicações de cada tema levam em consideração a sua aplicação imediata em um documento com formatação básica. Porém, todo o conteúdo pode ser aplicado conceitualmente ao documento atual e, no caso das figuras, tabelas, equações e itens, o conteúdo pode e deve ser aplicado em cada arquivo dos elementos de texto contidos neste projeto.

\section{Exemplo de Subseção  de Capítulo}
\label{cap01:sec02}
%pode apagar!
Para alterar este texto basta dar um 
\textbf{duplo clique “aqui”}, e o local de edição será aberto na sua caixa de edição do lado “esquerdo”.

%pode apagar tudo!
\textbf{Copie e cole seu texto aqui.}

A formatação usada  pelas Normas ABNTex2 são:

\textbf{Papel}: A4 – cor branca

\textbf{Fonte}: Times New Roman ou Arial- tamanho 12 – cor: preta. Nas citações com mais de 3 linhas, notas de rodapé, legendas e tabelas a fonte deve ter o tamanho 10.

\textbf{Itálico}: Deve ser usado nas palavras de outros idiomas. Esta orientação não se aplica às expressões latinas apud e et al.

\textbf{Margens}: Direita e inferior: 2cm / Esquerda e superior: 3cm
Parágrafos / Espaçamento: 1,5 entre linhas;

As referências devem ser separadas umas das outras com espaçamento duplo.

\textbf{Alinhamento do texto}: O texto do trabalho deve estar justificado para que fique alinhado às margens esquerda e direita. Esta formatação revela uma aparência mais organizada e o escrito fica melhor distribuído.